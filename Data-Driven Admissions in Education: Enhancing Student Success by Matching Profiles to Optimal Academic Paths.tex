\documentclass[conference]{IEEEtran}
\IEEEoverridecommandlockouts
% The preceding line is only needed to identify funding in the first footnote. If that is unneeded, please comment it out.
\usepackage{cite}
\usepackage{amsmath,amssymb,amsfonts}
\usepackage{algorithmic}
\usepackage{graphicx}
\usepackage{textcomp}
\usepackage{xcolor}
\usepackage{float}
%\restylefloat{table}
\def\BibTeX{{\rm B\kern-.05em{\sc i\kern-.025em b}\kern-.08em
    T\kern-.1667em\lower.7ex\hbox{E}\kern-.125emX}}
\begin{document}

\title{Data-Driven Admissions in Education: Enhancing Student Success by Matching Profiles to Optimal Academic Paths\\
{\footnotesize \textsuperscript{}}
\thanks{}
}

\author{\IEEEauthorblockN{Clément Combier}
\IEEEauthorblockA{\textit{Master 2 SIGLIS} \\
\textit{Université de Pau et des Pays de l'Adour}\\
Anglet, France \\
clement.combier@etud.univ-pau.fr}}
\maketitle

\begin{abstract}
In the wake of the COVID-19 pandemic and the release of the new \textit{baccalaureate} reform, French education authorities in higher studies faces a surge of enrolments and higher dropouts numbers. Higher grade from students in the baccalaureate as lead, the French registration system in place to accept more and more students in higher degrees paths. Sadly, these new reforms did not take into account the difficulty step created between secondary and higher studies. Thus augmenting the number of dropouts in students who don't have the capacity, motivation and/or will to continue in their path. 

We propose a solution to mitigate this dropout as well as helping academia to find \textit{excellence students} with compatible profile for a certain path (diploma and domain). Taking the problem at its root could lead to a \textit{two birds with one stone} resolution to the problem.

This paper focuses on critical issues within the education system and tries to differ a more holistic and personalized approach to student placement. By using data mining, analytic and machine learning, we hope to create a more harmonious and productive education landscape for both students and academic alike.
\end{abstract}
\vspace{8pt}
\begin{IEEEkeywords}
Higher education, Admission process, Machine learning, Data analytics, Success rate, Dropout rate, Student profile, Optimization, Profile-degree matching, Admission management, High-achieving students, Adaptive education.
\end{IEEEkeywords}
\vspace{16pt}

\section{Introduction}
Today education landscape in higher education is rapidly evolving due to higher levels of education and possibly COVID-19, the surge in higher education enrolment presents new challenges : how to manage increasing number of admissions and how to ensure academic success. 
An estimated 2.97 million students registering for higher education in France for the year 2021-2022, marking a 2.5\% increase from the previous year \cite{sous-direction_des_systemes_dinformation_et_des_etudes_statistiques_sies_les_2022}. Globally, UNESCO reported a rise from 28.5 million students in 1970 to an estimated 235 million in 2023 \cite{unesco_higher_2023}.

However, this increase in enrolment has not necessarily translated into higher success rates. As illustrated in Table \ref{tab:result_bachelor}, the cumulative success rate over 4 years for students enrolled in Bachelor's programs in France remains a concern, with only 39.8\% of the 2010 cohort succeeding by 2014.

\begin{table}[H]
    \centering
    \caption{Results in Bachelor's degree for the 2013 and 2014 sessions for students enrolled for the first time in the first year of Bachelor's in 2010-2011.\cite{kabla-langlois_fporsoc16b_ec2_enseignementpdf_2016}}
    \begin{tabular}{|c|c|}
        \hline
        \textbf{Headcount 2010} & \textbf{Cumulative success over 4 years (in \%)}\\
        \hline
        169 652 & 39.8 \\
        \hline
    \end{tabular}
    \label{tab:result_bachelor}
\end{table}

In France, the Parcour sup system was introduced to manage the influx of candidates and match them with suitable programs. Despite its intentions for uniformity and transparency, it has faced criticism for not adequately addressing the mismatch between student potential and program suitability, which is a contributing factor to the low success rates \cite{couto_parcoursup_2021}.

Enter the realm of machine learning and data analytics, technologies that promise a more nuanced and accurate approach to admissions. By analysing a myriad of factors beyond traditional metrics, this approach seeks to identify candidates whose profiles align perfectly with specific programs, thereby increasing the likelihood of academic success and addressing the mismatch issue.

This research endeavours to explore and validate the potential of machine learning and data analytics in revolutionizing the admission process. The motivation is twofold: to enhance the success rate of students by ensuring they are placed in programs where they are most likely to excel, and to reduce dropout rates by minimizing mismatches between students and programs.

This research will explore and proposes of different approach, starting with a detailed methodology, followed by a case study made witin the University of Pau et des Pays de l'Adour. 
We will then conclude, taking into account our findings and the result of our experiment 

\section{State of the art}
\label{sec:soa}
The body of literature concerning student dropout is comprehensive, encompassing a range of perspectives from diverse academic disciplines. The issue has been approached through psychological and economic lenses, utilizing both qualitative and quantitative research methods. This section is structured to reflect the various domains and methodologies present in prior studies. 

We've cut our literature analysis in two major subsections. The first one, \ref{subsec:soa_analyticalapproeach}, will review the different ways of analytical analysis on student dropout. These different approaches went from simple adaptation models to economic models and psycho-pedagogic models to predict and describe student dropout. Then, in \ref{subsec:soa_predictiveapproach} we are going to study different mathematical algorithms and models and use machine learning models to predict student dropout.

\subsection{Analytical approach}
\label{subsec:soa_analyticalapproeach}
This first part will delve inside the different perspectives of research outside the predictive fields using machine learning. We found these different point of view interesting to study for our future predictive model. These research have different conclusion on why students dropout from their studies, from an economical standpoint to a psychological one.
(Opazo et al. 2021)\cite{opazo_analysis_2021} have found an interesting link with the study from (Spady 1970) which found a correlation between the student dropout model and the social nature of suicide (\textit{social integration}), (Durkheim, 1951). This theory implies that the likelihood of suicide increase when there is an absence of : a)\textit{low normative congruence} and b)\textit{low friendship support}. Other literature from their state-of-the-art list these following as recurrent factors from multiple research\cite{opazo_analysis_2021}\cite{spady_dropouts_1970,tinto_dropout_1975,caspersen_teachers_2015,lidia_problema_2006,bejarano_caso_2017,sinchi_acceso_2018,cavero_voluntad_2011,velasco_alisis_nodate}: 
\begin{itemize}
\item family
\item previous educational background
\item academic potential
\item normative congruence
\item friendship support
\item intellectual development
\item educational performance
\item social integration
\item satisfaction
\item institutional commitment
\item student adaptation
\end{itemize}
We can decisively take into account these factors for our study has they have been proven to be recurrent factors throughout the literature on predicting students dropout.

\subsection{Predictive Approaches}
\label{subsec:soa_predictiveapproach}
Several researches focused on predictive approaches such as Naive Bayesian Algorithm, Association rules mining, ANN based algorithm, Logistic Regression, CART, C4.5, J48, Simple Logistic, JRip, Random Forest, Logistic regression analysis, ICRM2.\cite{mduma_survey_2019}. 
In this paper, we are going to analyse the following models \cite{mduma_survey_2019}\cite{quinlan_induction_1986,yadav_mining_2012,heredia_student_2015,ramirez_prediction_2018,cox_regression_1958,perez_modelo_2018,pandey_data_2011,hegde_higher_2018,cover_nearest_1967,mardolkar_forecasting_2020,zhang_neural_2000,rudin_stop_2019,siri_predicting_2015,m_alban_she_is_with_the_faculty_of_engineering_and_applied_sciences_of_the_technical_university_cotopaxi_neural_2019,cortes_support-vector_1995,boser_training_1992,cardona_predicting_2019,naicker_linear_2020,lee_machine_2019,behr_early_2020,friedman_stochastic_2002,eckert_analysis_2015,tenpipat_student_2020,liang_machine_2016,liang_big_2016,fischer_angulo_modelo_2012,miranda_analysis_2017,viloria_integration_2019,kemper_predicting_2020}:
\begin{enumerate}
\item Decision Trees
\item Logistic Regression
\item Naive Bayes
\item K-Nearest Neighbors (KNN)
\item Neural Networks
\item Support Vector Machine
\item Random Forest
\end{enumerate}
These models have been chosen as they have been the most recurring ones within the literature. We will analyse each models to list the pros and cons of each one and to determine which one(s) we should use to build our predictive system.

\subsubsection{Decision Trees}
\subsubsection{Logistic Regression}
\subsubsection{Naive Bayes}
\subsubsection{K-Nearest Neighbors (KNN)}
\subsubsection{Neural Networks}
\subsubsection{Support Vector Machine}
\subsubsection{Random Forest}
\subsubsection{Conclusion}


\section*{Acknowledgment}
\vspace{12pt}
\bibliographystyle{plain}
\bibliography{references}

\end{document}
