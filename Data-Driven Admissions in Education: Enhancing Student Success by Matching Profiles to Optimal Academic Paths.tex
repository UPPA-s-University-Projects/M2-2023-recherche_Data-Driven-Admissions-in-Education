\documentclass[conference]{IEEEtran}
\IEEEoverridecommandlockouts
% The preceding line is only needed to identify funding in the first footnote. If that is unneeded, please comment it out.
\usepackage{cite}
\usepackage{amsmath,amssymb,amsfonts}
\usepackage{algorithmic}
\usepackage{graphicx}
\usepackage{textcomp}
\usepackage{xcolor}
\def\BibTeX{{\rm B\kern-.05em{\sc i\kern-.025em b}\kern-.08em
    T\kern-.1667em\lower.7ex\hbox{E}\kern-.125emX}}
\begin{document}

\title{Data-Driven Admissions in Education: Enhancing Student Success by Matching Profiles to Optimal Academic Paths\\
{\footnotesize \textsuperscript{}}
\thanks{}
}

\author{\IEEEauthorblockN{Clément Combier}
\IEEEauthorblockA{\textit{Master 2 SIGLIS} \\
\textit{Université de Pau et des Pays de l'Adour}\\
Anglet, France \\
clement.combier@etud.univ-pau.fr}}
\maketitle

\begin{abstract}
In the wake of the COVID-19 pandemic and the release of the new \textit{baccalaureate} reform, French education authorities in higher studies faces a surge of enrolments and higher dropouts numbers. Higher grade from students in the baccalaureate as lead, the French registration system in place to accept more and more students in higher degrees paths. Sadly, these new reforms did not take into account the difficulty step created between secondary and higher studies. Thus augmenting the number of dropouts in students who don't have the capacity, motivation and/or will to continue in their path. 

We propose a solution to mitigate this dropout as well as helping academia to find \textit{excellence students} with compatible profile for a certain path (diploma and domain). Taking the problem at its root could lead to a \textit{two birds with one stone} resolution to the problem.

This research endeavours to address a critical issue within the education system, offering a more holistic and personalized approach to student placement. By reimagining the criteria for admission, we aspire to create a more harmonious and productive educational landscape for both students and academic institutions. 
\end{abstract}
\vspace{8pt}
\begin{IEEEkeywords}
Higher education, Admission process, Machine learning, Data analytics, Success rate, Dropout rate, Student profile, Optimization, Profile-degree matching, Admission management, High-achieving students, Adaptive education.
\end{IEEEkeywords}
\vspace{16pt}

\section{Introduction}
In today's rapidly evolving educational landscape, optimizing the admission process in higher education institutions has never been more critical. With the aftermath of the COVID-19 pandemic and the surge in students seeking higher education, universities face the daunting task of managing a vast number of admissions while ensuring they admit candidates best suited for their programs. It is estimated that for the year 2021-2022, 2.97 millions students have registered for higher education.\cite{sous-direction_des_systemes_dinformation_et_des_etudes_statistiques_sies_les_2022} 

The Parcoursup system, introduced as a centralized platform for higher education admissions in France, aimed to streamline this process. While it brought some level of uniformity and transparency, criticisms have arisen regarding its limitations in truly identifying candidates' aptitude and ensuring a match between a student's potential and the chosen program.

\begin{table}
    \centering
    \caption{Results in Bachelor's degree for the 2013 and 2014 sessions for students enrolled for the first time in the first year of Bachelor's in 2010-2011.\cite{kabla-langlois_fporsoc16b_ec2_enseignementpdf_2016}}
    \begin{tabular}{|c|c|}
        \hline
         \textbf{Headcount 2010} & \textbf{Réussite cumulée en 4 ans (en \%)} \\
         \hline
         169 652 & 39.8 \\
         \hline
    \end{tabular}
    
    \label{tab:my_label}
\end{table}

Enter the realm of machine learning and data analytics, technologies that promise a more nuanced and accurate approach to admissions. By analyzing a myriad of factors beyond traditional metrics, this approach seeks to identify candidates whose profiles align perfectly with specific programs, thereby increasing the likelihood of academic success.

This research endeavors to explore and validate the potential of machine learning and data analytics in revolutionizing the admission process. The motivation is twofold: to enhance the success rate of students by ensuring they are placed in programs where they are most likely to excel and to reduce dropout rates by minimizing mismatches between students and programs.

This paper will delve into the intricacies of the proposed approach, starting with a detailed methodology, followed by case studies from the University of Pau and Pays de l’Adour. The results section will present findings, and the discussion will explore implications, potential challenges, and future directions.

In the broader context of higher education, this research holds significant promise in addressing some of the pressing challenges faced by universities today. By reading this paper, stakeholders in the education sector will gain insights into a novel approach that not only promises better student outcomes but also paves the way for a more efficient and effective admission process.


\section*{Acknowledgment}
\vspace{12pt}
\bibliographystyle{plain}
\bibliography{references}

\end{document}
