\documentclass[conference]{IEEEtran}
\IEEEoverridecommandlockouts

%Loading packages
\usepackage{cite}
\usepackage{amsmath,amssymb,amsfonts}
\usepackage{algorithmic}
\usepackage{graphicx}
\usepackage{textcomp}
\usepackage{xcolor}
\usepackage{float}
\usepackage{subfiles}
\usepackage[toc]{glossaries}

%Style coding
%\restylefloat{table}
\def\BibTeX{{\rm B\kern-.05em{\sc i\kern-.025em b}\kern-.08em
    T\kern-.1667em\lower.7ex\hbox{E}\kern-.125emX}}

%Load glossary
\loadglsentries{res/glossaries/abbreviation}


\begin{document}

\title{Data-Driven Admissions in Education: Enhancing Student Success by Matching Profiles to Optimal Academic Paths\\
{\footnotesize \textsuperscript{}}
\thanks{}
}

\author{\IEEEauthorblockN{Clément Combier}
\IEEEauthorblockA{\textit{Master 2 SIGLIS} \\
\textit{Université de Pau et des Pays de l'Adour}\\
Anglet, France \\
clement.combier@etud.univ-pau.fr}}
\maketitle
\thispagestyle{plain}
\pagestyle{plain}

\tableofcontents
\listoffigures
\listoftables
\printglossary[type=\acronymtype, title=Accronymes, toctitle=Accronymes]

\vspace{16pt}
\begin{abstract}
In the wake of the COVID-19 pandemic and the release of the new \textit{baccalaureate} reform, French education authorities in higher studies faces a surge of enrolments and higher dropouts numbers. Higher grade from students in the baccalaureate as lead, the French registration system in place to accept more and more students in higher degrees paths. Sadly, these new reforms did not take into account the difficulty step created between secondary and higher studies. Thus augmenting the number of dropouts in students who don't have the capacity, motivation and/or will to continue in their path. 

We propose a solution to mitigate this dropout as well as helping academia to find \textit{excellence students} with compatible profile for a certain path (diploma and domain). Taking the problem at its root could lead to a \textit{two birds with one stone} resolution to the problem.

This paper focuses on critical issues within the education system and tries to differ a more holistic and personalized approach to student placement. By using data mining, analytic and machine learning, we hope to create a more harmonious and productive education landscape for both students and academic alike.
\end{abstract}
\vspace{8pt}

\begin{IEEEkeywords}
Higher education, Admission process, Machine learning, Data analytics, Success rate, Dropout rate, Student profile, Optimization, Profile-degree matching, Admission management, High-achieving students, Adaptive education.
\end{IEEEkeywords}
\vspace{16pt}

\section{Introduction}
\label{sec:introduction}  
\subfile{sections/introduction}

\section{State of the art}
\label{sec:soa}
\subfile{sections/soa}


\section{Analysis}
\label{sec:analysis}
\subfile{sections/analysis}


\section{Conceptual implementation}
\label{sec:conceptualanalysis}
\subfile{sections/conceptual_proposal}

\vspace{16pt}
\section{Acknowledgment}
I would like to thank all the faculty for their welcome, help, and support throughout this year. Helping me both on my paper and on my work at the lab. I would like to personally thanks Dr. Ernesto Exposito, Dr. Mamadou Lamine Gueye and Dr. Houssam Kansso as well as PhD. student Nicolas Evain for their help and welcome. A special thanks go to Dr. Mamadou Lamine Gueye for following me as a tutor for this last year of Master.

I would also like to thank the entirety of the 2023-2024 SIGLIS and Industry 4.0 promotion with whom I've been able to share great times, knowledge and some fun moments during the year. A special thanks go to the SIGLIS promotion for supporting me for two year, especially my flatmates Dorian Cazabat, Lise Laville and Gabriel Das Neves Rodrigues. 

Finlay, I would like to thank my friends and family for their support, their motivation and for being here in the time of need.
\vspace{12pt}

\bibliographystyle{plain}
\bibliography{bib/references}
\printglossary

\end{document}
