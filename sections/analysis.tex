\documentclass[../main.tex]{subfiles}
\graphicspath{{\subfix{../res/}}}
\begin{document}
We will now do an analysis from the literature review on how we can approach the problem, using what's been already made and how we can improve on it.

\subsection{Factors}
\label{subsec:analysis_factors}
First of all, what differentiate this research from all the other we have read throughout the literature analysis is that we are not seeking prediction on student's dropout but rather on student success and early in the process and not during the curriculum year. However, there is plenty of interesting information we can gather from these papers. As described in the \ref{sec:soa} State of Art, we can gather factors that, in theory could help predict student's dropout. We can hypothesize that by using these factors to determine if one student is at risk of dropping-out, it could for another predict its success in a specific formation. From the list of factors we were able to gather, we have made a statistical analysis of the frequency they appear and their overall score within each paper they are mention it. Below, the table from this study concluding our research.


This part of the study showed us how difficult it was to examine our feeding data and how to define the output we are thriving for. Factors, whether they are inputs or outputs, will strongly depend on multiple factors like :
\begin{itemize}
    \item Societal norm
    \item Economical norm
    \item Institution
    \item Definition of excellence (some factors may depend on the definition of themselves)
    \item Time frame (all factors above will change over time)
\end{itemize}

The human part (sociological part) is the most complex in this research. As human have evolved and will evolve, norms will change, and specific factors now may differ in the future. 
The goal will not to find the correct and universal combination of factors, but rather give a framework of \acrshort{ml} algorithm and how to feed them depending on the need of the institution and the goal they are thriving towards.

\subsection{Machine Learning algorithm}
\label{subsec:analysis_mlalgo}
Secondly, we need to understand which algorithm model have been used the most and which present the best outcome for our need. As for the factors, we can extrapolate the problem and take it in reverse. So by learning which algorithm presents the best result to predict student's dropout, we could hypothesize that they could also be used to detect student's success. 

\subsection{Analysis conclusion}
\label{subsec:analysis_conclusion}
Both our hypothesis and result must now be verified by providing a methodology and using a test dataset to send to our pipeline in order to feed our machines.
We may find that one or both hypothesis are not correct and we will need to restudy factors and machine learning algorithm to answer our need and problematic. 
In the next part, \ref{sec:conceptualanalysis} Conceptual implementation, we are going to present our methodology and workflow. Explaining the reasons for our choice of factors and algorithm as well as presenting our entire pipeline for our system.
\end{document}