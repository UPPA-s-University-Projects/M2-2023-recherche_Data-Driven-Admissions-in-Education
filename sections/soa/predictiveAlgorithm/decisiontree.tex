\documentclass[../../../main.tex]{subfiles}
\graphicspath{{\subfix{../../../res/}}}
\begin{document}
The Decision Tree model, in ML, stand out as a fundamental and versatile algorithm. It is widely recognized for their simplicity and efficacy for classification and regression tasks alike.
Their tree-like structure gave them their name to the algorithm, comprised of nodes and branches (symbolizing decisions and their possible consequences). This helpful visualization and interpretation makes complex decision-making processes easier and intuitive for the user. This visualization is really powerful and is an invaluable tool in various applications, ranging from data mining to advanced research in AI.

Decision trees have been used in a wide range of fields such the medical, game-theory weather prediction and much more.\cite{quinlan_induction_1986}.
From the literature, decision tree seems to have a precision of around 83\% based on multiple papers (73\% \cite{viloria_integration_2019}, 87.27\% \cite{ramirez_prediction_2018}, 83\%\cite{kemper_predicting_2020} and around 90\% \cite{tenpipat_student_2020}). This research paper has evaluated multiple algorithm within decision tree and graphed out a ROC Curve for each of their algorithm. In a similar domain as this study, which is students dropout in South Korea schools and how to predict them. Their ROC curves showed that for a low rate of false positive, we get an outstanding true positive rate. Approximately attaining their limits around 0.2 false positive rate\cite{lee_machine_2019}. This study shows the predictive efficiency of the decision tree model. 
One downside of each study is the granularity of student prediction. They have shown which variable can be used to determine which kind of students may be at risk of dropout, but they lack a micro vision to alert staff about a specific student at risk.
\end{document}