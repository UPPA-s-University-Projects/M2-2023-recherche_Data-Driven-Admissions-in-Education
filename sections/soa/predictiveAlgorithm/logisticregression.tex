\documentclass[../../../main.tex]{subfiles}
\graphicspath{{\subfix{../../../res/}}}
\begin{document}
Logistic regression, different from linear regression by its ability to handle binary outcomes (scenarios where the outcome is dichotomous), is an important part of the statistical method in the field of \acrshort{ml}. It is particularly interesting for its role in classification tasks. This method works by modeling the probability of a binary response, based on one or more predicator variables (independent variables). 
Our scenario of student dropout is not dichotomous by nature. However, this method can be a first step for data cleansing and classification of at risk student and out of risk students.

An interesting approach by this study \cite{lan_sparse_2014} - which change the granularity of the prediction to a student based one - is to create a “grade book” (a source of information commonly used in the context of classical test theory \cite{novick_axioms_1966}), filed with ones if the student answer correctly to question \textit{i} or 0 if not. Then, weights are added to each question depending on their difficulty to define if a student is at risk of failing and dropout. However, this approach does include some sort of e-learning method, and does solve the problem from a per-student level. Yet, it does not take into account other factors cited above to paint a bigger picture of a life of a student, and thus, defining if this student is at risk or not. It could simply be failing this course, and even if a pattern arises for a specific student (of failing), can we conclude he is at risk of dropping out?
\end{document}
