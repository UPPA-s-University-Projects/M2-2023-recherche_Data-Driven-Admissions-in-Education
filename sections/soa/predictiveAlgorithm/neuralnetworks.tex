\documentclass[../../../main.tex]{subfiles}
\graphicspath{{\subfix{../../../res/}}}
\begin{document}
\acrfull{nn} envision the \acrfull{ai} and \acrshort{ml} landscape by taking inspiration from the human brain's structure and function. A network is composed of layers of interconnected neurons, each capable of performing simple probability computations. Thanks to these layers of computations, a Neural Network can learn complex patterns and relationships in data from the input. This makes the incredibly powerful for a wide range of tasks. 
This technique turned into a field of its own, which now includes different structured \acrfull{nn} such as \acrfull{cnn} and \acrfull{rnn}, each useful for specific types of dataset and tasks. Their learning capacity and ability make them an incredible powerful tool in the field of \acrshort{ai} and \acrshort{ml}. 

\acrfull{nn} is a vast field in itself. This first study used the Perceptron Multilayer algorithm with an admissible error of 0.0001 and using the hyperbolic tangent function for activating input and output layers\cite{viloria_integration_2019}. They have estimated that the classification was 83\% accurate according to the \acrshort{roc} curve of the neural network classifiers.
\begin{table}[H]
    \centering
    \caption{Comparison of the classifier evaluation parameters\cite{viloria_integration_2019}}
    \begin{tabular}{|c|c|c|}
        \hline
        \textbf{Model name} & \textbf{Precision}  & \textbf{Recall}\\
        \hline
        Decision Tree & 72\% & 64\% \\
        \hline
        \acrlong{nn} & 73\% & 65\% \\
        \hline
    \end{tabular}
    \label{tab:comparaison_classifier_eval_param_viloria}
\end{table}

This new table of comparison of classifier per model shows us one more time that we cannot establish if one is better than the other or not. Once more, each model has is particularities, advantages, and disadvantages. 

The perceptron Multilayer algorithm is the most commonly used algorithm within the field of \acrshort{ai} and \acrshort{ml}.\cite{siri_predicting_2015} For this research, they have study :
\begin{quote}
    "This study asked two research questions: 
    \begin{enumerate}
        \item How accurately do pre-entry students’ characteristics predict the risk of dropout?
        \item Which characteristics weigh most in predicting the risk of dropout?"
    \end{enumerate} 
\end{quote}\cite{siri_predicting_2015}
For both question, they have concluded that the model is a pilot instrument and may be enhanced. They advised that :
\begin{quote}
    "Future research should consider the extension of the study to other groups of students not only enrolled in the healthcare area because it could reveal the importance of including additional variables not considered in this research."
\end{quote}\cite{siri_predicting_2015}

In this other study \cite{siri_predicting_2015}, their neural network model had a prediction accuracy of about 84\% for their first study group, 81\% for group 2 and 76\% for the dropout group. 

We will see in the conclusion of this literature review, which model seems to be the more fitted for our problem, taking into account the complexity and cost of each model as a system.
\end{document}