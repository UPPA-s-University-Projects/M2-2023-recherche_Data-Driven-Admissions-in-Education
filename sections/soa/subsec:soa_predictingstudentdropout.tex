\documentclass[../../main.tex]{subfiles}
\graphicspath{{\subfix{../../res/}}}
\begin{document}
The body of literature concerning student dropout is comprehensive, encompassing a range of perspectives from diverse academic disciplines. The issue has been approached through psychological and economic lenses, utilizing both qualitative and quantitative research methods. This section is structured to reflect the various domains and methodologies present in prior studies.

We've cut this section in two major subsections. The first one, \ref{subsubsec:soa_analyticalapproach}, will review the different ways of analytical analysis on student dropout. These different approaches went from simple adaptation models to economic models and psycho-pedagogic models to predict and describe student dropout. Then, in \ref{subsec:soa_predictiveapproach} we are going to study different mathematical algorithms and models and use ML models to predict student dropout.

It is interesting to look at the opposite side of the "problem" to gather which information have been used to predict student dropout. We could then hypothesise and extrapolate about if for one student it could predict its failure and for another its success.
However, some factors cited are sensibly comprehensible about one's failure (for example, it is logical that a student with health and family problem will have a greater probability of failure.) However, even with a good situation, one can be uninterested in its formation of choice and then dropout. 
What we want to filter and understand from the literature is factors that could lead us to the potential intellect of student such as their hobbies, past experience in the domain, motivation.

But first, let's define success and what is a student choice. There is many ways of defining what is student success. From a societal or governmental (administrative) point of view, the definition can vary technically, but tell the same thing nonetheless. For institution, student's success can be define as the percentage of student who got their diploma, in how many years and the overall grade score. For governments and administration, it could be define as the success rate (number of graduation and overall grades) of national exam. As for the student point of view, success is getting their diploma by the end of the usual cursus. Some may evaluate their success on the grades they've got as well, which can be a factor to determine students who are motivated not only by just getting their diploma but also in actually succeeding their formation by getting good grades and acceding to the top of the promotion.

A common recurrent topic on that subject is the lack of holistic approach by academia and researchers on the concept of student success. It is often regarded as a “common sense” where student success is measured not by the score of students, but rather how many got their diplomas. \cite{weatherton_success_2021}.
\end{document}