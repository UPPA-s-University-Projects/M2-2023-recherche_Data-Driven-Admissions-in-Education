
\documentclass[../../main.tex]{subfiles}
\graphicspath{{\subfix{../../res/}}}
\begin{document}
This first part will delve inside the different perspectives of research outside the predictive fields using ML. We found these different point of view interesting to study for our future predictive model. These research have different conclusion on why students dropout from their studies, from an economical standpoint to a psychological one.\cite{opazo_analysis_2021} have found an interesting link with the study from (Spady 1970) which found a correlation between the student dropout model and the social nature of suicide (\textit{social integration}), (Durkheim, 1951). This theory implies that the likelihood of suicide increase when there is an absence of : a)\textit{low normative congruence} and b)\textit{low friendship support}. Other literature from their state-of-the-art list these following as recurrent factors from multiple research\cite{opazo_analysis_2021,spady_dropouts_1970,tinto_dropout_1975,caspersen_teachers_2015,lidia_problema_2006,bejarano_caso_2017,sinchi_acceso_2018,cavero_voluntad_2011,velasco_alisis_nodate}: 
\begin{itemize}
    \item family
    \item previous educational background
    \item academic potential
    \item normative congruence
    \item friendship support
    \item intellectual development
    \item educational performance
    \item social integration
    \item satisfaction
    \item institutional commitment
    \item student adaptation
\end{itemize}
We can decisively take into account these factors for our study has they have been proven to be recurrent factors throughout the literature on predicting students dropout.

In another study based in South Korea, they have defined the other type of factor for students (high-school students) \cite{lee_machine_2019} :
\begin{itemize}
    \item Diseases
    \item Family Problem
    \item Poor Academic Performance
    \item Poor Relationship with Others
    \item Strict School Rules
\end{itemize}

\end{document}