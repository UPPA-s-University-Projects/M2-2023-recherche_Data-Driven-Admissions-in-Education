\documentclass[../../main.tex]{subfiles}
\graphicspath{{\subfix{../../res/}}}
\begin{document}
From different literature review and papers, research seems to diverge and agree on different aspects. Making it difficult to pinpoint a specific variable that could point us in the direction of success prediction. From the country to institution itself, student history and background. It seems like choice has an important effect on whether a student may be successful in its study or not. However, we cannot satisfyingly say that choice makes student success.  Some outliers may have bad grades but be in reality a high potential success student. It all depends on a myriad of factors, which can be distilled into different variable that our ML algorithms can learn from. We have to actually see and search (as we are going to do when looking at student dropout factors), which factors can be used to feed our models in order to correctly predict student success and find these outliers within the mass.\cite{kuh_what_2006, sa_how_2018}
We found these different point of view interesting to study for our future predictive model. These research have different conclusion on why students dropout from their studies, from an economical standpoint to a psychological one.\cite{opazo_analysis_2021} have found an interesting link with the study from\cite{spady_dropouts_1970} which found a correlation between the student dropout model and the social nature of suicide (\textit{social integration}), \cite{durkheim_suicide_1951}. This theory implies that the likelihood of suicide increase when there is an absence of : a)\textit{low normative congruence} and b)\textit{low friendship support}. Other literature from their state-of-the-art list these following as recurrent factors from multiple research\cite{opazo_analysis_2021,tinto_dropout_1975,caspersen_teachers_2015,lidia_problema_2006,bejarano_caso_2017,sinchi_acceso_2018,cavero_voluntad_2011,velasco_alisis_nodate}: 
\begin{itemize}
    \item Family
    \item Previous educational background
    \item Academic potential
    \item Normative congruence
    \item Friendship support
    \item Intellectual development
    \item Educational performance
    \item Social integration
    \item Satisfaction
    \item Institutional commitment
    \item Student adaptation
\end{itemize}
We can decisively take into account these factors for our study has they have been proven to be recurrent factors throughout the literature on predicting students dropout.

In another study based in South Korea, they have defined the other type of factor for students (high-school students) \cite{lee_machine_2019} :
\begin{itemize}
    \item Diseases
    \item Family Problem
    \item Poor Academic Performance
    \item Poor Relationship with Others
    \item Strict School Rules
\end{itemize}
However, this cannot directly be used to determine success as it is clear these factors are only looking into the negative background of the student. It is highly probably that a sick student with family troubles will have harder time getting success in its studies than an healthy student. We could extrapolate and find linked factors that could determine success, but for the simplicity of our research we are going to let down of these factors, only retaining the school rules. It would be a good parameter to set our machines to depending on how strict we want to be with the registration. Some institution are easier to access than end elite-like ones.

\end{document}