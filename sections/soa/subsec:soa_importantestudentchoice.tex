\documentclass[../../main.tex]{subfiles}
\graphicspath{{\subfix{../../res/}}}
\begin{document}
Our first question we have to ask is if student choice really matter for the success in higher education or not. Does other factors influence this rate, does student choice influence heavily the chance of success for a student?
Multiple paper have searched the reasons behind student success to highlight the important factor to take into account.

A common recurrent topic on that subject is the lack of holistic approach by academia and researchers on the concept of student success. It is often regarded as a “common sense” where student success is measured not by the score of students, but rather how many got their diplomas. \cite{weatherton_success_2021}.

From different literature review and papers, research seems to diverge and agree on different aspects. Making it difficult to pinpoint a specific variable that could point us in the direction of success prediction. From the country to institution itself, student history and background. It seems like choice has an important effect on whether a student may be successful in its study or not. However, we cannot satisfyingly say that choice makes student success.  Some outliers may have bad grades but be in reality a high potential success student. It all depends on a myriad of factors, which can be distilled into different variable that our ML algorithms can learn from. However, we cannot use only student's choice. We have to actually see and search (as we are going to do when looking at student dropout factors), which factors can be used to feed our models in order to correctly predict student success and find these outliers within the mass.\cite{kuh_what_2006, sa_how_2018}
\end{document}