\documentclass[../main.tex]{subfiles}
\graphicspath{{\subfix{../res/}}}
\begin{document}
Today education landscape in higher education is rapidly evolving due to higher levels of education and possibly COVID-19, the surge in higher education enrolment presents new challenges : how to manage increasing number of admissions and how to ensure academic success. 
An estimated 2.97 million students registering for higher education in France for the year 2021-2022, marking a 2.5\% increase from the previous year \cite{sous-direction_des_systemes_dinformation_et_des_etudes_statistiques_sies_les_2022}. Globally, UNESCO reported a rise from 28.5 million students in 1970 to an estimated 235 million in 2023 \cite{unesco_higher_2023}.

However, this increase in enrolment has not necessarily translated into higher success rates. As illustrated in Table \ref{tab:result_bachelor}, the cumulative success rate over 4 years for students enrolled in Bachelor's programs in France remains a concern, with only 39.8\% of the 2010 cohort succeeding by 2014.

\begin{table}[H]
    \centering
    \caption{Results in Bachelor's degree for the 2013 and 2014 sessions for students enrolled for the first time in the first year of Bachelor's in 2010-2011.\cite{kabla-langlois_insee_2016}}
    \begin{tabular}{|c|c|}
        \hline
        \textbf{Headcount 2010} & \textbf{Cumulative success over 4 years (in \%)}\\
        \hline
        169 652 & 39.8 \\
        \hline
    \end{tabular}
    \label{tab:result_bachelor}
\end{table}

In France, the Parcoursup system was introduced to manage the influx of candidates and match them with suitable programs. Despite its intentions for uniformity and transparency, it has faced criticism for not adequately addressing the mismatch between student potential and program suitability, which is a contributing factor to the low success rates \cite{couto_parcoursup_2021}.

But what can be categorized as student success, and how to tell if a student enters the profile of excellence? Most of the time, success is, for institute, the number of student who graduate their degree. \cite{weatherton_success_2021}. However, even if this correlation can indicate a good rate of success for an institute or university, is it really indicative of real success? For our research, we have chosen to stay with the simple success definition of how many students can graduate to stay within a binary model for our ML. However, to really answer the problematic of highlighting an excellent student, we will take into account other factors.
Another definition we have to pin is the definition of excellent student. By definition, it could be written like so, “an excellent student is one with good grades, a good understanding of the concepts and a general interest in the field of study.” Independent of students with ease for learning, an excellent student may not perform well in a casual course cursus, but out stand in a specific field he or she is interested in. 

This research endeavours to explore and validate the potential of ML and data analytic in revolutionizing the admission process. The motivation is twofold: to enhance the success rate of students by ensuring they are placed in programs where they are most likely to excel, and to reduce dropout rates by minimizing mismatches between students and programs. We also thrive to found more excellence students within the mass of registration.

We are going to base our experiments and result on the french academic system. However, we are going from the principal that any academic system could use this research to build such registration helping systems. Because of a lack of literature on the french system, we have extended it to the entire world, including all different academic system from different countries. Because the academic process matters less than the actcual need and hope from a student and institution point of view, we can exploit these different datas for our research. Yet, as researcher and as readers, we suggest that a line been drawn and remember that a big part of this system is the culture of the country in which this system is based. We are using universal factors to feed our system, but some may vary from country to country.

Another point we need to clarify is that this research is not made to discriminate student nor help the "elite" by creating an even bigger chasm in societal problematic. It is in fact a way to reduce this gap and give each student a chance of getting into higher education and earn some sort of diploma that will suit their need and hope.

This research will explore and proposes of different approach, starting with a detailed methodology, followed by a case study made within the University of Pau et des Pays de l'Adour. 
We will then conclude, taking into account our findings and the result of our experiment 
\end{document}